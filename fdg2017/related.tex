
Cardona-Rivera and Young~\cite{cardona2014games}
detailed a conceptual framework following the slogan {\em games as
conversation}, grounding the communicative strategies of games in cognitive
science for human-to-human conversational understanding, such as Grice's
maxims~\cite{grice1975logic}. They offer a linguistic and semiotic approach to
understanding how a game communicates affordances (possibilities for
action) to a player.  For an account of the game's half of the equation,
which includes the visual, textual, and audio feedback mechanisms intended
to be processed by the player, this application of linguistics, psychology,
and design seems appropriate, much like the study of cinematic language for
film.  On the other hand, we argue that a PL approach better supports
understanding of the player-to-game direction, since the language the
player speaks toward a digital game is formal and unambiguous.

Researchers have previously recognized the value in formalizing interaction
vocabularies, realizing certain interaction conventions as a {\em single}
``video game description language''~\cite{ebner2013towards} whose
implementation as VGDL~\cite{schaul2013video} has been used in game AI research. We
suggest instead that the design space of player languages is as varied as
the design space of programming languages and herein give an account of
what it would mean to treat each language individually.
Our project suggests that an appropriately expressive computational
framework analogous to VGDL should be one that can accommodate the encoding
of many such languages, such as a {\em meta-logical framework} like the
Twelf system for encoding and analyzing programming language
designs~\cite{pfenning1999system}. 

% AI action languages (planning, event calc, etc.); general game AI frameworks
Any investigation into formalizing actions within an interactive system
shares ideas with ``action languages'' in AI extending as far back as
McCarthy's situation calculus~\cite{mccarthy1969some} and including
planning languages and process calculi.  These systems have been studied in the
context of game design, e.g. the Ludocore system~\cite{smith2010ludocore};
however, AI researchers are mainly interested in these formalisms as internal
representations for intelligent systems and the extent to which they support
reasoning.  Conversely, we are interested their potential to support player
expression and facilitate human-computer conversation.

% PlaySpecs?

Hazel

FRP? Other PL stuff; embedded DSLs?
