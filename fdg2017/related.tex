
Cardona-Rivera and Young~\cite{cardona2014games}
detailed a conceptual framework following the slogan {\em games as
conversation}, grounding the communicative strategies of games in cognitive
science for human-to-human conversational understanding, such as Grice's
maxims~\cite{grice1975logic}. They offer a linguistic and semiotic approach to
understanding how a game communicates affordances (possibilities for
action) to a player.  For an account of the game's half of the equation,
which includes the visual, textual, and audio feedback mechanisms intended
to be processed by the player, this application of linguistics, psychology,
and design seems appropriate, much like the study of cinematic language for
film.  On the other hand, we argue that a PL approach better supports
understanding of the player-to-game direction, since the language the
player speaks toward a digital game is formal and unambiguous.

Researchers have previously recognized the value in formalizing interaction
vocabularies, realizing certain interaction conventions as a {\em single}
``video game description language''~\cite{ebner2013towards} whose
implementation as VGDL~\cite{schaul2013video} has been used in game AI research. We
suggest instead that the design space of player languages is as varied as
the design space of programming languages and herein give an account of
what it would mean to treat each language individually.
Our project suggests that an appropriately expressive computational
framework analogous to VGDL should be one that can accommodate the encoding
of many such languages, such as a {\em meta-logical framework} like the
Twelf system for encoding and analyzing programming language
designs~\cite{pfenning1999system}. 

% AI action languages (planning, event calc, etc.); general game AI frameworks
Any investigation into formalizing actions within an interactive system
shares ideas with ``action languages'' in AI extending as far back as
McCarthy's situation calculus~\cite{mccarthy1969some} and including
planning languages and process calculi.  These systems have been studied in the
context of game design, e.g. the Ludocore system~\cite{smith2010ludocore};
however, AI researchers are mainly interested in these formalisms as internal
representations for intelligent systems and the extent to which they support
reasoning.  Conversely, we are interested their potential to support player
expression and facilitate human-computer conversation.

Some theoretical and experimental investigations have been carried out
about differences between game interfaces along specific axes, such as
whether the interface is ``integrated'' (or one might say diagetic), versus
extrinsic to the game world in the form of menus and
buttons~\cite{llanos2011players, jorgensen2013gameworld}. These
investigations suggest an interest in more detailed and formal ontologies
of game interfaces, which our work aims to provide.

(XXX transition)
Hazelnut is a formal model of a program editor that enforces that
every edit state is meaningful (it consists of a well-defined syntax
tree, with a well-defined type)~\cite{omar17hazelnut}.
%
Its type system and editing semantics permit \emph{partial programs},
which contain missing pieces and well-marked type inconsistencies.
%
Specifically, Hazelnut proposes a \emph{editing language}, which
defines how a cursor moves and edits the syntax tree; the planned
benefits of this model range from better editing assistance, the
potential to better automate systematic edits, and further
context-aware assistance and automation based on statistical analysis
of (semantically-rich) corpses of recorded past edits, which consist
of \emph{traces} from this language~\cite{omar17hazel}.
%
Likewise, in the context of game design, we propose a player language;
we expect similar benefits, as outlined elsewhere in the paper (??).

\paragraph{Functional reactive programming.}
%
Functional reactive programming (FRP) describes computations that
consume and produce time-varying data using techniques and ideas from
functional programming~
\citep{ElliottHu97,WanHu00,Cooper06embeddingdynamic,Krishnaswami11,Krishnaswami13,Czaplicki2013AFR}.


