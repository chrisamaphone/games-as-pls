\newcommand{\OmitThis}[1]{}

Consider the right-hand-side of Figure~\ref{fig:gameloop}: The game
parses and interprets an unambiguous syntax of player intent, which
either advances the game state (as shown in the figure), or the game
state cannot advance as the player intended, which the game somehow
signals to the player (not shown).

We model a \emph{response} to the move-take player intent with the
following BNF definition, of~$\mathit{resp}$:
%
  \begin{eqnarray*}
  \mathit{resp} &::=& \mathsf{success} \mid \mathsf{failure}
  \end{eqnarray*}
%
Figure~\ref{fig:gameloop} shows the case where the player intent of
``take flask'' (formally, the syntax ``$\ctake$'') leads to the game
world performing this intent as a successful action, and responding
accordingly with ``Taken.''
%
Formally, we model this $\mathit{resp}$ as~``$\mathsf{success}$,'' as
defined above.
%
Likewise, if the flask cannot be taken (e.g., the flask is not near
the player, or is already in the player's possession, etc.), the game
responds with~$\mathsf{failure}$.

As with the player's intent, which may exist as both raw input
\emph{and} as formal syntax, each formal response can be conveyed as
raw output in a variety of ways (e.g., as textual words, pictures, or
sounds).
%
In real games there are often two two levels of game-to-player
feedback:
%
Feedback through the game world, and feedback \emph{outside} the game
world (e.g., a pop-up message with an error, guidance or advice).
%
For simplicity, move-take gives feedback outside of the game state,
e.g., as pop-up messages.

% XXX Mention semiotics anywhere?

\newcommand{\GameStep}[4]{
 \left< #1; #2 \right>
 \longrightarrow 
 \left< #3; #4 \right>
}

To capture the formal relationship between player intent as syntax,
and game response as syntax, we introduce a four-place
\emph{game-step} relation:
\[
\GameStep{G_1}{\mathit{intent}}{G_2}{\mathit{resp}}
\]
This relation formalizes the dynamic behavior of the right-hand-side
of Figure~\ref{fig:gameloop}.  It consists of four parts: An initial
game state~$G_1$, a player intent~$\mathit{intent}$, a resulting game
state~$G_2$ and a game response~$\mathit{resp}$.

As as standard in PL formalisms, we give the rules that define this
relation as inductive \emph{inference rules}, which can each be read
as a logical inference.  That is, given evidence for the premises on
the top of the rule, we may conclude the bottom of the rule.
%
For instance, here are two example rules:
\[
\infer{
  \GameStep{G_1}{\ctake}{G_2}{\textsf{success}}
}{ 
  G_1 \vdash \textrm{player near}~\textsf{flask}
  \quad
  \quad
  \textsf{playerTake}(G_1,\textsf{flask}) \equiv G_2
}
\]
\[
\infer{
  \GameStep{G}{\ctake}{G}{\textsf{failure}}
}{
  G \vdash \textsf{not}\big( \textrm{player near}~\textsf{flask} \big)
}
\]
The first rule formalizes the case shown in the RHS of
Figure~\ref{fig:gameloop}.
%
The second rule formalizes the opposite outcome, where the flask
cannot be taken.
%
Notably, the first rule has two premises: To be taken by the player,
it suffices to show that in the current game state~$G_1$, the flask is
near the player (first premise), and that there exists an advanced
game state~$G_2$ that results from the player taking this flask~(second premise).
%
In the second rule, there is only one game state~$G$, since the flask
cannot be taken and consequently, the game state does not change.

Like the syntax of player intent and game responses, these rules are
also unambiguous.
%
Consequently, we view these rules as a mathematical definition with an
associated strategy for constructing formal (and informal) proofs
about the game mechanics.

For instance, we can formally state and attempt to prove that for all
player intents~$\mathit{intent}$ and game worlds, $G_1$, there exists
a corresponding game world~$G_2$ and game response~$\mathit{resp}$.
%
That is, the statement of the following conjecture:
\[
\begin{array}{ccc}
\forall G_1, \mathit{intent}.
&
\exists G_2, \mathit{resp}.
&\GameStep
  {G_1}{\mathit{intent}}
  {G_2}{\mathit{resp}}
\end{array}
\]
Using standard PL techniques, the proof of this conjecture gives rise
to an abstract algorithm that implements the game mechanics by
analyzing each possible case for the current game-state and player
intent.
%
Indeed, this is precisely the reasoning required to show that an
implementation of the game is \emph{complete} (i.e., there exists no
state and input that will lead the game into an undefined situation).

Reasoning about this completeness involves reasoning about when each
rule is applicable.  
%
For instance, the rule for a successful player intent of~\textsf{take}
requires two premises:
$G_1 \vdash \textrm{player near}~\textsf{flask}$
and
$\textsf{playerTake}(G_1,\textsf{flask}) \equiv G_2$.
%
The first is a \emph{logical judgment} about the game world involving
the proposition~$\textrm{player near}~\textsf{flask}$, which may or
may not be true, but which is computable.
%
The second is a \emph{semantic function} that transforms a game state
into one where the player takes a given object; in general, this
function may be undefined, e.g., if the arguments do not meet the
functions \emph{pre-conditions}.
%
For instance, a precondition of the
function~$\textsf{playerTake}(-,-)$ may be that the object is not
already in the possession of the player.
%
In this case, to show that the game mechanics do not ``get stuck'', we
must show that~$\textrm{player near}~\textsf{flask}$ implies that the
player does not already posses the flask.
%
(Otherwise, we should add another rule to handle the case that the
player intends to take the flask but already possess it).
%

The design process for programming language semantics often consists of
trying to write examples and prove theorems, and revising the language
definition to account for what one learns during this design process.
%
In the next section, we refine intents and semantics further by
introducing the notion of a \emph{context}.


\OmitThis{
\section{Precise player-game dynamics via operational semantics}
\label{sec:opsem}

\newcommand{\Player}[4]{\ensuremath{{#1}{:}\left<{#2},{#3},{#4}\right>}}
\newcommand{\PropIsTrue}[2]{\ensuremath{#1 \vdash #2~\textsf{true}}}
\newcommand{\StepsTo}[4]{\ensuremath{\left<{#1};{#2}\right> \longrightarrow \left<{#3};{#4}\right>}}

\section{Precise player-game dynamics via operational semantics}
\label{sec:opsem}

%The player consists of a 

%The operational semantics employs notions of the game and player states.

%The following BNF definitions give meta variable conventions (the left-hand-side variable names) and syntactic cases (the right-hand-side of each rule).

\begin{figure*}
\small

\begin{tabular}{l|l}

\begin{minipage}{0.68\textwidth}
\begin{tabular}{lccll}
unique IDs & $i,j$ & $::=$ & $\cdots$ & (\emph{abstract}, e.g., numbers, symbols, noun phrases)
\\
propositions   & $P$ & $::=$ & $\cdots$ & Game world propositions~(See~$\PropIsTrue{G}{P}$)
\\
player command & $C$ & $::=$ & $\cdot$ $~|~$ $c$ & Unit and atomic commands (0 and 1 turns, resp.)
\\
               &     & $~|~$ & $C_1~{;}~C_2$ & Command sequences (multi-turn commands)
\\
               &     & $~|~$ & $\textsf{until}~P~\textsf{do}~C$ & Command loops (e.g., for player skills)
\\
atomic command & $c$ & $::=$ & $\textsf{grasp}~i$ & Attempt to place object~$i$ into player's hand
\\
               &     & $|$   & $\textsf{drop}$ & Drop the held object, if any
\\
               &     & $|$   & $\textsf{giveTo}~i$ & Give held object to another~(named~$i$)
\\
               &     & $|$   & $\textsf{takeFrom}~i$ & Take hold of object from another~(named~$i$)
\\
               &     & $|$   & $\textsf{moveTo}~i$ & Move adjacent to object, place or agent~(named~$i$)
\\
               &     & $|$   & $\textsf{moveThrough}~i$ & Move through opening~(named~$i$)
\\[5mm]
\end{tabular}
\begin{tabular}{lccll}
game state & $G$ & $::=$ & $\cdots$ & (\emph{abstract})
\\
player state & $p$ & $::=$ & $\Player{i}{B}{h}{C}$ & Player identity~$i$, bag~$B$, hand~$h$ and commands~$C$
\\
bag state (spaces) & $B,S$ & $::=$ & $\cdot~|~s::S$ & Zero or more unit spaces
\\
hand state (unit space) & $h,s$ & $::=$ & $\square~|~i$ & An empty space, or an object~(named~$i$)
\\
\end{tabular}
\end{minipage}

&

\begin{minipage}{0.3\textwidth}
\[
\begin{array}{ll}
\fbox{$G_1 \equiv G_2$}~\textrm{World equivalence}
\\
\fbox{$B_1 \equiv B_2$}~\textrm{Bag equivalence}
\\
\fbox{$\PropIsTrue{G}{P}$} 
\\
\textrm{In world~$G$, proposition~$P$ is true}.
\\[2mm]
\fbox{$\StepsTo{G}{p}{G'}{p'}$}
\\[2mm]
\infer{  
  \StepsTo
      {G}{ 
        \Player{i}{B_1}{\square}{\textsf{grasp}~j}
      }
      {G}{ 
        \Player{i}{B_2}{j}{\cdot}        
      }
}
{
  B_1 \equiv j :: B_2
}
\\[2mm]
\infer{  
  \StepsTo
      {G}{ 
        \Player{i}{B_1}{j}{\textsf{grasp}~k}
      }
      {G}{ 
        \Player{i}{B_3}{j}{k}{\cdot}
      }
}
{
  B_1 \equiv k :: B_2
  \\
  B_3 \equiv j :: B_2
}
\\[2mm]
\infer{  
  \StepsTo
      {G}{ 
        \Player{i}{B_1}{j}{\textsf{grasp}~k}
      }
      {G}{ 
        \Player{i}{B_3}{j}{k}{\cdot}
      }
}
{
  B_1 \not\equiv k :: B_2
  \\
  B_3 \not\equiv j :: B_2
}
\end{array}
\]
\end{minipage}


\end{tabular}

\[
\begin{array}{l}
\\[2mm]
\infer{  
  \StepsTo
      {G}{ 
        \Player{i}{B}{k}{\textsf{grasp}~k}
      }
      {G}{ 
        \Player{i}{B}{k}{\textsf{errMsg}~\textrm{``you are already holding $k$''}}
      }
}
{
\quad
}
\\[2mm]
\infer{
  \StepsTo
      {G_1}{
        \Player{i}{B_1}{j}{\textsf{drop}}
      }
      {G_2}{
        \Player{i}{B_3}{\square}{\cdot}
      }
}
{
  G_2 \equiv \textsf{dropItem}(G_1, j, \textsf{pos}(i))
}
\\[2mm]
\infer{
  \StepsTo
      {G_1}{
        \Player{i}{B}{k}{\textsf{giveTo}~j}
      }
      {G_2}{
        \Player{i}{B}{\square}{\cdot}
      }
}
{
  G_1\,{\vdash}\,\textsf{isNear}(i, j)
  \\
  G_2 \equiv \textsf{giveItemTo}(G_1, k, j)
}
\end{array}
\]


\caption{Definitions for an operational semantics: Captures precise
  player-game dynamics for a primitive adventure game.}
\end{figure*}
}
