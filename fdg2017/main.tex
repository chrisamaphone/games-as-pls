\documentclass[sigconf]{acmart}

\usepackage{booktabs} % For formal tables


% Copyright
%\setcopyright{none}
%\setcopyright{acmcopyright}
%\setcopyright{acmlicensed}
% \setcopyright{rightsretained}
%\setcopyright{usgov}
%\setcopyright{usgovmixed}
%\setcopyright{cagov}
%\setcopyright{cagovmixed}


% DOI
% \acmDOI{10.475/123_4}

% ISBN
% \acmISBN{123-4567-24-567/08/06}

%Conference
% \acmConference[WOODSTOCK'97]{ACM Woodstock conference}{July 1997}{El
%  Paso, Texas USA} 
% \acmYear{1997}
% \copyrightyear{2016}

% \acmPrice{15.00}


\begin{document}
\title{Languages of Play}
\subtitle{Towards semantic foundations for playable software}


\author{Chris Martens}
\affiliation{%
  \institution{North Carolina State University}
  \city{Raleigh} 
  \state{NC} 
}
\email{martens@csc.ncsu.edu}


\begin{abstract}
Abstract goes here. A full paper's maximum length is 10 pages; a short
paper's is 6 pages. Track: probably Game Design and Development.
\end{abstract}


\keywords{games, programming languages, formal methods}

\maketitle

\section{Introduction}

\section{Related Work}

\section{A Framework}

  Syntax, type systems, and operational semantics

  \subsection{Player affordances as syntax}

  Additive vs. subtractive affordances

  \subsection{Structured affordances as type systems}

  \subsection{Mechanics as operational semantics}

  \subsection{Game environment as external runtime}

  \subsection{Play traces as (normalized) programs}
  
  Argument for having a syntactically-well-founded structured term for a
  play trace
  
\section{Example}

Stardew Valley

\section{Player skills as programs}

And skill-building as iterative program construction

Mention BOTS

Nondeterminism, protocols

\section{Discussion}

  What this enables:
  \begin{itemize}
  \item Scripting languages for games, for free
  \item Co-creative interfaces and collaborative play, a la MUDs/ZZT
  \item Reasoning about/analyzing games and possible skill trees
  \end{itemize}

  A single ``video game description language'' is a misleading direction to
  take a formal understanding of games, because games as expressive forms
  are as diverse as programming languages.

  Instead, we can imagine meta-frameworks c.f. logical frameworks for
  encoding PLs and specifying their meaning.

  Future work: try to draw further analogies. Better REPLs for PLs? 

\section{Conclusion}

  Summary of contributions (new ideas, why they matter)

\begin{acks}
  acknowledgements
\end{acks}

% \bibliographystyle{ACM-Reference-Format}
% \bibliography{sigproc} 

\end{document}
