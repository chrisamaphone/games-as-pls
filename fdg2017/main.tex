\documentclass[sigconf]{acmart}

\usepackage{booktabs} % For formal tables


% Copyright
%\setcopyright{none}
%\setcopyright{acmcopyright}
%\setcopyright{acmlicensed}
% \setcopyright{rightsretained}
%\setcopyright{usgov}
%\setcopyright{usgovmixed}
%\setcopyright{cagov}
%\setcopyright{cagovmixed}


% DOI
% \acmDOI{10.475/123_4}

% ISBN
% \acmISBN{123-4567-24-567/08/06}

%Conference
% \acmConference[WOODSTOCK'97]{ACM Woodstock conference}{July 1997}{El
%  Paso, Texas USA} 
% \acmYear{1997}
% \copyrightyear{2016}

% \acmPrice{15.00}


\begin{document}
\title{Languages of Play}
\subtitle{Towards semantic foundations for game interfaces}


\author{Chris Martens}
\affiliation{%
  \institution{North Carolina State University}
  \city{Raleigh} 
  \state{NC} 
}
\email{martens@csc.ncsu.edu}

\author{Matthew Hammer}
\affiliation{%
  \institution{University of Colorado, Boulder}
  \city{Boulder}
  \state{CO}
}
\email{XXX@XXX.edu}


\begin{abstract}
Abstract goes here. A full paper's maximum length is 10 pages; a short
paper's is 6 pages. Track: probably Game Design and Development.
\end{abstract}


\keywords{games, programming languages, formal methods}

\maketitle

\section{Introduction}

The field of player modeling, and the general study of the way in which
players traverse a game space, examines the interaction between the rules
of a system and the choices made by the player. The argument usually
follows that we need to construct detailed models of player cognition: 
while a game as a self-contained entity can allow us to learn about its
mechanics and properties as a formal system, we cannot understand the {\em
dynamics} of that system unless we also account for the human half of the
equation.

What often goes missing in such a discussion, however, is the {\em
interface} between the two. Cardona-Rivera and Young (XXX cite) proposed
the conceptual framework of {\em games as conversation}, emphasizing that
play is a two-way street: a game provides affordances for action, a player
takes action based on those affordances, and the game responds in turn with
feedback that develops the player's mental model for the behavior
of the system implied by the interaction as a whole. And, just as the field
of film studies has developed a notion of {\em vocabulary} that visual
storytellers use to communicate with audiences, scholars have studied the
vocabulary and representational conventions of games (XXX cite something re
games literacy?) that designers employ to communicate with players, e.g. a
horizontal green bar floating above an avatar representing its health, or:
a crack in an object typically understood as a wall may indicate that it
can be broken to access another part of the game.

We have recognized that a symbolic, discretized and representational
approach to looking at design language has value. But since games are
interactive, it would seem that we should account for another component:
what is the analog to {\em design language} employed by the player? How
does the player communicate their intent, and how does a digital game
recognize this intent?

Clearly they use some kind of language, but since the constraints on what
language they may use are wholly determined by a piece of software (the
game interface), we argue that this language has more in common with a {\em
programming language} (loosely defined as a formal language whose meaning
is fully grounded in a computational system) than a natural language. We
argue that conceptualzing player communication in terms of hardware
interactions (button presses, mouse clicks) is too {\em low-level} of an
approach to meaningfully study interaction, and that instead we should make
sense of these interactions on a {\em per-game} basis in terms of the {\em
meaning} with which a game imbues these interactions. 

Accordingly, each game provides a player with a unique formal language,
termed ``programming language'' (though it may have a great deal in common
with other games' languages) to express themselves. While some researchers
have recognized interaction vocabularies common to certain types of games,
their conclusion was to codify these conventions as a {\em single}
programming language (VGDL, XXX cite), misleadingly positioned as though it
could represent all (or a broad category) of games in a high-level way.
We suggest that instead, the landscape of game representations and player
affordances is as varied as the landscape of programming languages, and an
adequate computational framework should be one that can accommodate the
encoding of many such languages, such as a {\em logical framework} (XXX
cite Twelf or something).

What is to be gained by the game-interfaces-as-programming languages analogy?

The programming languages (PL) community has a long and deep history of
assigning mathematically formal semantics to languages and analyzing those
semantics. As games researchers become more interested in the emergent
consequences of the systems they assemble, the tools of PL theory have a
lot to offer. For example, PL theory provides an account of how to relate
meaning in a {\em low level} representation (such as assembly language, or
on the games side, hardware-level interactions) to meaning in a {\em
high-level} language (such as a language like Java or a game's state
space), allowing us to draw abstraction boundaries and consider system
components in compositional terms. In Salen and Zimmerman (XXX cite) terms,
PL theory can help us formalize the {\em constituative mechanics} of a
game's interface so that we may consider distinct {\em operational rules} 
that interface with them. 

(XXX example: parser vs. hypertext interactive fiction?)

Furthermore, by considering a player's language of expression as an object
of study in its own right, we center her as a co-designer of the experience
afforded by a game. When we treat a player's interactions as not simply an
arbitrary sequence of button presses that advances and reveals the
designer's intent, but instead as its own distinct {\em voice} that a game
system must listen and respond to, we enable the player to {\em co-create}
with the system, potentially developing stronger systems thinking skills in
the player, eliciting deeper understanding and even deeper emotional
investment.

In the remainder of the paper, we concretize the analogy by introducing the
components of a theoretical account of a programming language (syntax, type
system, and operational semantics) and its analog in a game. We walk
through an example to illustrate how to think about a game in terms of its
language-like affordances, and we demonstrate the payoff of this line of
thought by extending the metaphor to account for {\em player skills} as
``programs,'' effectively representing mental models for how to accomplish
some task in terms of the game's linguistic constructs.


\section{Related Work}

\section{A Framework}

  (that table from the doc should go here)

  Syntax, type systems, and operational semantics

  \subsection{Player affordances as syntax}

  Additive vs. subtractive affordances

  \subsection{Structured affordances as type systems}

  \subsection{Mechanics as operational semantics}

  \subsection{Game environment as external runtime}

  \subsection{Play traces as (normalized) programs}
  
  Argument for having a syntactically-well-founded structured term for a
  play trace

  
\section{Example}

Stardew Valley

\section{Player skills as programs}

And skill-building as iterative program construction

Mention BOTS

Nondeterminism, protocols

\section{Discussion}

  What this enables:
  \begin{itemize}
  \item Scripting languages for games, for free
  \item Co-creative interfaces and collaborative play, a la MUDs/ZZT
  \item Reasoning about/analyzing games and possible skill trees
  \end{itemize}

  A single ``video game description language'' is a misleading direction to
  take a formal understanding of games, because games as expressive forms
  are as diverse as programming languages.

  Instead, we can imagine meta-frameworks c.f. logical frameworks for
  encoding PLs and specifying their meaning.

  Future work: try to draw further analogies. Better REPLs for PLs? 

\section{Conclusion}

  Summary of contributions (new ideas, why they matter)

\begin{acks}
  acknowledgements
\end{acks}

% \bibliographystyle{ACM-Reference-Format}
% \bibliography{sigproc} 

\end{document}
