Up to this point, we have described languages of player expression as
single utterances in isolation from one another, but we have not considered
their composition, i.e. how sequences of actions can be employed to carry out
more complex tasks. Games with rich player action languages afford modes of
exploratory and creative play: consider item crafting in
Minecraft~\cite{minecraft}, puzzle solving in Zork~\cite{blank1980zork}, or
creating sustainable autonomous systems like a supply chain in
Factorio~\cite{factorio}, a farm in Stardew Valley~\cite{stardew}, or a
transit system in Mini Metro~\cite{minimetro}.  Each of these activities
asks the player to understand a complex system and construct multi-step
sequences of actions to accomplish specific tasks, such as going from the
game state of having a few coins and an empty field to the game state of
having a thriving farm.

A language, as we have formalized it, gives us the atomic pieces from which
we can construct these sequences, like Lego bricks. {\em Compositionality}
in language design is the principle that we may understand the meaning and
behavior of compound structures (e.g. sequences) in terms of the meaning
and behavior of each of its pieces (e.g. actions), together with the
meaning of how they are combined (e.g. carried out one after the other, or
in parallel). In this section, we describe how we might make sense of {\em
player skills} in terms of complex programs written in the player language.

Such programs might be integrated into a game's mechanics so that a player
explicitly writes such programs, as in the BOTS game, an interactive
programming tutor that asks players to write small imperative programs that
direct an avatar within a virtual world~\cite{hicks2012creation}, or Cube
Composer\footnote{\url{http://david-peter.de/cube-composer/}}, in which
players write functional programs to solve puzzles. However, for now, we
primarily intend this account of player skills as a conceptual tool.

\subsection{Example: Stardew Valley}
\newcommand{\param}[1]{\langle #1 \rangle}
\newcommand{\syn}[1]{\mathsf{#1}}

Our initial $\{\cmove, \ctake\}$ example is too simple to craft really
compelling examples of complex programs, so here we examine
Stardew Valley and its game language for the sake of considering player
skills. In Stardew Valley, the player has an inventory that permits varied
interactions with the world, beginning a number of tools for farming (axe,
hoe, scythe, pickaxe) which do different things in contact with the
resources in the surrounding environment; most include extracting some
resource (wood, stone, fiber, and so on), which themselves enter the
player's inventory and can be used in further interaction with the game
world. There are also context-sensitive interactions between the player and
non-player characters (NPCs), interfaces through which new items may be
purchased (shops), and mini-games including fishing (fish may also be sold
at high value).

While a full account of the language that this game affords the player is
beyond the scope of this paper, we include a representative sample of the
actions and affordances found in this game that may be used to construct
player skills.

Stardew Valley uses keyboard keys (WASD) for movement within and between
world ``rooms'' as well as point-and-click actions for selecting items in
one's inventory and interacting with in-room entities. The player's avatar
is shown on-screen, moved by WASD. It must be near an entity for the player
to interact with it. They can then either $\syn{apply}$ the
currently-selected inventory item to the in-world entity with a left-button
click, or right-button click, which does something based on the entity
type, e.g. doors and chests open, characters speak, and collectible items
transfer to the player's inventory. We refer to this last action as
$\syn{inquire}$. We also note that, for the sake of our example, 
movement towards an entity and movement offscreen (towards another room)
are the only meaningful and distinct types of movement, which we refer to
as $syn{move\_near}$ and $\syn{move\_offscreen}$. In BNF form, the syntax
is:


% (an item is something held, an entity is something in the world)
\begin{eqnarray*}
intent &::=& \syn{select} \param{item}\\
       &\mid& \syn{apply} \param{entity}\\
       &\mid& \syn{inquire} \param{entity}\\
       &\mid& \syn{move\_near} \param{entity}\\
       &\mid& \syn{move\_offscreen} \param{direction}
\end{eqnarray*}

We leave the definition of items and entities abstract, but we could
imagine it to simply list all possible items and entities in the world as
terminals.
From these atomic inputs, we can start to construct higher-level actions
performed in the game most frequently---tilling land,
planting seeds, conversing with NPCs, and so on. These blocks of code may
be assigned names like functions to be called in many contexts:

% we want to accurately reflect the
% {\em extensibility} of Stardew Valley's player language by describing the
% general system from which these actions are derived. In other words, rather
% than having each instance of such an action be a special case that a player
% must learn how to speak as an independent vocabulary term, instead, they
% learn it through {\em composing} the pre-existing constructs of selecting items
% and applying them to objects in the world. For example: (XXX explain this
% more)
% 
% XXX semantics?

Higher-level actions:
\begin{verbatim}
action till = 
  select hoe; move_near hard_ground; 
  apply hard_ground
action plant = 
  select seeds; move_near tilled_ground; 
  apply tilled_ground
action mine = 
  select pickaxe; move_near rock; apply rock
action enter_shop = 
  move_near shop; inquire door(shop)
action talk = move_near npc; inquire npc
\end{verbatim}

XXX transition


\subsection{Combining actions into skills}

The task of defining a language of compound actions enters novel territory
in programming language design, because compound player actions
are typically interleaved with game response: the player does not have full
knowledge of how the game world works, so she experiments with its
affordances, develops a model for which actions map to which responses, and
then plans accordingly. Furthermore, even a complete model of the game
world might include some nondeterminism, such as ``half the time, fishing
gives me a fish, but half the time I only pull up trash.'' Players
therefore often create ad-hoc plans on the basis of real-time game
responses, gradually learning to form complex plans by thinking further in
advance about their consequences.

This is one of the ways games are said to teach us {\em systems thinking}:
by showing us, piece by piece, what each part of a system accomplishes in
isolation, then framing one's activity within an over-arching goal, the
player must reason about her actions' effects on the world and how they
interact with one another, not just how they behave in isolation. We
observe the cause-and-effect behavior and start to form {\em higher-level
plans} in terms of the skills we learn how to do: instead of {\em plant
crop; water crop; harvest crop; sell crop} we may refer to the collective
action as {\em farming} and incorporate this action with other high-level
skills (mining, fishing) into a plan for how our character should spend her
day.

One possible plan she can take is to scavenge the local wildlife: after
using the scythe on enough wild brush, she may find wild seeds for free,
which may be planted. Another option is to purchase some inexpensive seeds
at the general store. Then she must learn to grow the crop: tilling earth,
optionally fertilizing it, placing seeds in the ground, and then watering
it day after day (in between which other tasks may be accomplished).
Finally she must harvest the crop and take it to market to sell, then
repeat the process with a stronger financial foundation.

XXX continue

\subsection{Stardew Valley Bots}

Examples:

Farming a crop:
\begin{verbatim}

fun water_until_harvestable[t](p: planted(t))
: crop(t)
=
do try_harvest(p)
      recv <result: crop(t) + growing(t)>.
        case result of
        c:crop(t) => c
      | g:growing(t) => water(g, w); wait(day); try_harvest(g)

fun grow_crop[t : croptype](s:soil, w:watering_can)
: crop(t)
=
  do
    get_seeds(t) || till_soil(s)
  recv <s: seeds(t), g: tilled_soil>.
    do
      plant(s, g)
    recv <p: planted(t)>.try_harvest(p)
        
\end{verbatim}

Fishing:
\begin{verbatim}
fun get_fish(r:rod, w:water, nf:notfishing) 
: rod * water * notfishing * fish
=
  do  
    go_fish(r,w,nf)
  recv <r:rod, w:water, ft : (fish*notfishing)+(trash*notfishing)>.
    case ft of
      inl <f, nf> => <r, w, nf, f>
    | inr <t, nf> => do dispose t recv (). get_fish(r,w,nf)
\end{verbatim}


