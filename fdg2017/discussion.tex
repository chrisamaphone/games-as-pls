%% Possible things to include:
% - Staged syntax - how to capture? (e.g. "take..." "take what?" "ball"
%   or Prom Week's 2 step interaction formation (select actor, then select
%   who they are acting on))
% - Context-dependent vs. acontextual syntax and its correspondence with
%   the interface stack (hardware inputs vs. linguistic intent - actions
%   performed repeatedly by the player vs. strategies applied to specific
%   pieces of the environment)

Having established a vocabulary of syntax and semantics for player
languages, we can now revisit the potential benefits of this account
mentioned in the introduction and discuss them in more detail.

\subsection{Compositionality for play traces and player skills}

One of the major things that a programming language account provides is
{\em compositionality}: a system for making sense of meaning of a complex
artifact in terms of the meaning of its pieces. This comes up in two places
for looking at games:

{\bf Structured play traces.} With a formalized game language, the sequence
of steps along the transition system described by the operational semantics
forms a mathematical artifact that is subject to deeper analysis than what
can be gained simply from screen or input device recordings. For example,
we can carry out causal analysis, asking ``why'' queries of trace data,
e.g. ``Why was the player to unlock the door before defeating the boss?'',
as well as filtering traces for desired properties: ``Show me a play trace
where the player used something other than the torch to light the room.''
The recent PlaySpecs project (XXX cite) suggests interest in formulating
traces this way to support this kind of query.

{\bf Player skills as programs.} While a play trace may be interpreted as a
straight-line program, even more interesting is the idea of latent
structure in player actions, such as composing multiple low-level game
actions into a higher-level skill, following the cognitive idea of
``chunking.'' We map this idea onto that of {\em functions} in the
programming language that take arguments, generalizing over state space
possibilities (e.g.: the red key opened the red door, so for all colors
$C$, a $C$ key will open a $C$ door). Further reasoning forms like case
analysis to handle unpredictable game behavior and repeating an action
until a condition holds are also naturally expressed as programming
language constructs.


\subsection{Abstraction boundaries between input and mechanics}

(XXX provide evidence, and or more detail?)

Formalizing game interfaces gives us the tools to explore alternative
interfaces to the same underlying mechanics, without needing to port
game logic between different graphical interface frameworks. For example,
the interactive fiction community has been exploring alternatives to the
two traditional dichotomy of ``parser vs. hypertext'' for presenting
text-based games and interactive story-worlds. An abstraction boundary
between the underlying mechanics, map, and narrative of the world, and the
view and input mechanisms used to interact with it, could open the doors to
research on user interfaces that support players' mental models of a world
conveyed in text.

\subsection{Scripting languages for free and co-creative play}

Finally, a strong advantage of (XXX)


\subsection{Future Work}
  % Future work: try to draw further analogies. Better REPLs for PLs? 
  In future work, we intend to build software for realizing game language
  designs and experimenting with protocol-based game AI developed as
  programs in these languages. In another direction, we aim to innovate in
  PL design outside of games, such as read-eval-print loops (REPLs) for
  live programming that includes rapid feedback loops motivated by
  gameplay, as well as distributed and concurrent systems that may benefit
  from the protocol-based approach proposed here.

