To test the extent to which this analogy holds beyond a smallest example,
we attempt to describe the logic of a complex resource economy game with
an extensive player language. In {\em Stardew Valley} (XXX cite), the
player has an inventory that permits varied interactions with the world,
beginning a number of tools for farming (axe, hoe, scythe,
pickaxe) which do different things in contact with the resources in the
surrounding environment; most include extracting some resource (wood,
stone, fiber, and so on), which themselves enter the player's inventory and
can be used in further interaction with the game world. There are also
context-sensitive interactions between the player and non-player characters
(NPCs), interfaces through which new items may be purchased (shops), and
mini-games including fishing.

While a full account of the language that this game affords the player is
beyond the scope of this paper, what follows is an attempt to include a
representative sample of the actions and affordances found in this game,
presented in terms of the framework described above.

(XXX summarize what we are doing in this section)

\subsection{Syntax}

% move, apply tool, interact (open chests, talk to people, open doors,
% sleep)
% eating, giving things to npcs

(XXX concrete vs. abstract syntax analog to controls vs. player actions?)

\newcommand{\param}[1]{\langle #1 \rangle}
\newcommand{\syn}[1]{\mathsf{#1}}

(an item is something held, an entity is something in the world)
\begin{eqnarray*}
action &::=& \syn{select} \param{item}\\
       &\mid& \syn{apply} \param{entity}\\
       &\mid& \syn{inquire} \param{entity}\\
       &\mid& \syn{move\_near} \param{entity}\\
       &\mid& \syn{move\_offscreen} \param{direction}
\end{eqnarray*}

While it would be tempting to use the syntax layer to encode the kind of
higher-level actions performed in the game most frequently---tilling land,
planting seeds, conversing with NPCs---we want to accurately reflect the
{\em extensibility} of Stardew Valley's player language by describing the
general system from which these actions are derived. In other words, rather
than having each instance of such an action be a special case that a player
must learn how to speak as an independent vocabulary term, instead, they
learn it through {\em composing} the pre-existing constructs of selecting items
and applying them to objects in the world. For example: (XXX explain this
more)

Higher-level actions:
\begin{verbatim}
action hoe = select hoe; move_near shrub; apply shrub
action mine = select pickaxe; move_near rock; apply rock
action enter_shop = move_near shop; inquire door(shop)
action talk = move_near npc; inquire npc
action plant = select seeds; move_near tilled_ground; apply tilled_ground
\end{verbatim}

\subsection{Context Dependence}

include discussion of whether the above sequences of actions will actually
succeed or not; protocols...

include discussion of inventory menu, shops, crafting, recipes, other contextual menus
(XXX here or later?)

\subsection{Semantics}

(We can probably just describe these in prose rather than writing out the
syntax in full)

XXX define predicates like extracts, cost of crop, etc?

% apply tool depends on the tool and the thing you apply it to;
% mining yields stone, copper
% chopping tree yields wood (nondeterministic: seeds, sap, acorns, etc.)
Case $\syn{apply}\param{entity}$:\\
If the player is near an object that can be {\em extracted} with the
applied tool (e.g. a tree may be extracted by an axe; stone and ores may be
extracted by the pickaxe), compute and display a state change removing that object and
replacing it with its drops. \\
If the player is not near such an object, compute and display nothing.

XXX say something about drops being random

The game has some internal rules that will detect collisions between the
player and dropped items to add them to her inventory (these collision
rules are not part of the player language, since they occur passively).

Case $\syn{move}\param{direction}$:\\
Change the player location (XXX mention above caveat re collisions)


Case $\syn{inquire}\param{object}$:\\
XXX

crops yield money

money can buy things in stores

recipe menu

interact depends on whether a person, chest, door, other game object


